\documentclass[a4paper,12pt]{article} 

%paquetes
\usepackage{graphicx}
\usepackage[spanish]{babel} 
\usepackage[utf8]{inputenc}
\usepackage{textcomp}
\usepackage{amssymb}
\usepackage{float}
\usepackage{subfig}
\usepackage{listings}

%caracteristicas de paginas
\pdfpagewidth 8.5in
\pdfpageheight 11in
\setlength\oddsidemargin{-0,21in}
\setlength\evensidemargin{-0,21in}
\setlength\topmargin{-2cm}
\setlength\textwidth{7in}
\setlength\textheight{9in}
\setlength\parskip{0.1in}

%%%%%%%%%%%%%%%%%%%%%%%%%%%%%%%%%%%%%%%%%%%%%%%%%%%%%%%%%%%%%

\begin{document} 

\title{Introducci\'on a la soluci\'on num\'erica de ODE's \\
\large Gu\'ia computacional 1 - Mec\'anica Cl\'asica 2016 - Clase G. Mindlin}
\author{Ignacio Poggi - L.U: 567/07 - ignaciop.3@gmail.com}
\maketitle

%\tableofcontents  %lo ponemos para trabajo largos que necesiten indice


\section{Enunciado}

En la secci\'on de materiales adicionales de la c\'atedra se encuentra un programa principal y un integrador Runge-Kutta de orden 4 (rk4) en lenguaje C. En el programa principal se encuentra escrita una ecuaci\'on diferencial a integrar, los par\'ametros y las condiciones iniciales. Sobre este c\'odigo van a trabajar en las siguientes actividades realizando las modificaciones pertinentes para su problema en particular.

En ubuntu es posible compilar y ejecutar el c\'odigo directamente desde una terminal abierta en una carpeta que contenga tanto el programa principal como el integrador rk4:\newline

\framebox[1.1\width]{gcc ODE\_ejX.c -o ode\_ejX -lm rk4.c}

\framebox[1.1\width]{./ode\_ejX}\newline

Se obtendr\'a como salida un archivo llamado $ejX.dat$ , donde $X$ es el n\'umero del ejercicio, que contiene el resultado de la integraci\'on. Los resultados pueden ser analizados graficamente mediante un graficador, en nuestro caso utilizaremos \textbf{gnuplot} que se controla mediante comandos en terminal.\newline


{\Large \textbf{Actividad 1}}

Editar el codigo de ODE.c para analizar los siguientes puntos:

\begin{itemize}
\item C\'omo var\'ia el resultado seg\'un el paso de integraci\'on. Programe una integraci\'on con el m\'etodo de Euler y compare.
\item Analizar c\'omo evoluciona el sistema dadas distintas condiciones iniciales.
\end{itemize}
Qu\'e tipo de conclusiones puede obtener a partir de los an\'alisis anteriormente realizados.\newline


{\Large \textbf{Actividad 2 - Oscilador arm\'onico amortiguado}}

El oscilador arm\'onico amotiguado es un problema del cual se conoce la soluci\'on anal\'itica cuya ecuaci\'on diferencial que rige el movimiento es:

\begin{equation}
	\frac{d^2 x}{dt^2} + 2\gamma\frac{dx}{dt} + \omega_0^2x = 0
\end{equation}

Estudie num\'ericamente las soluciones del sistema seg\'un la relaci\'on de los par\'ametros, para ello: escriba la ecuaci\'on de segundo orden como dos ecuaciones de primer orden, var\'ie $\gamma$ y $\omega$ e integre. Tambi\'en analice distintas condiciones iniciales. Compare con lo conocido de la soluci\'on anal\'itica, para ello grafique como evoluciona la posici\'on en el tiempo, la velocidad y cu\'al es la trayectoria en el espacio de fases $x  \ddot{x}$.\newline


{\Large \textbf{Actividad 3 - Oscilador de Van der Pol}}

Es un tipo de oscilador con un amortiguamiento no lineal descripto a principio de siglo por Van der Pol quien estudi\'o circuitos el\'ectricos con componenten no lineales obteniendo la ecuaci\'on:

\begin{equation}
	\frac{d^2 x}{dt^2} + \mu(x^2-1)\frac{dx}{dt} + x = 0
\end{equation}

Este sistema presenta soluciones oscilatorias para ciertos valores del par\'ametro $\mu$ que son conocidad como oscilaciones de relajaci\'on. Esta ecuaci\'on tiene una importancia en la ciencia ya que fue usada en distintos campos para describir por ejemplo, el comportamiento de una falla tect\'onica o el potencial de acci\'on de una neurona. Esto se debe a que el sistema seg\'un los valores de $x$ presenta un amortiguamiento positivo (como el de la actividad 2 donde el sistema pierde energ\'ia), y para otros presenta un amortiguamiento "negativo" donde el sistema gana energ\'ia. Esto produce que eventualmente la energ\'ia perdida en un ciclo sea igual a la ganada generando oscilaciones autosostenidas. Este sistema se ver\'a con m\'as detalle avanzado el curso, en esta pr\'actica se propone realizar un acercamiento de forma num\'erica para tener cierta comprensi\'on de c\'omo se comporta el mismo.

\begin{itemize}
\item Escriba el sistema como dos ecuaciones de primer orden.
\item Inspecione num\'ericamente las soluciones posibles del sistema, estudie como var\'ian seg\'un la variaci\'on del par\'ametro $\mu$. Para ello grafique la trayectoria $x$ en funci\'on del tiempo, la velocidad $\ddot{x}$ en funci\'on del tiempo y tambi\'en el espacio de fases $x  \ddot{x}$.
\item Modifique tambi\'en las condiciones iniciales y estudie num\'ericamente las respuestas del sistema. Para ello grafique la trayectoria $x$ en funci\'on del tiempo, la velocidad $\ddot{x}$ en funci\'on del tiempo y tambi\'en el espacio de fases $x  \ddot{x}$.
\end{itemize}


{\Large \textbf{A entregar}}

Se deber\'a entregar un trabajo de la actividad 2 y 3, con los c\'odigos, los gr\'aficos obtenidos para las integraciones num\'ericas propuestas y el correspondiente an\'alisis para cada caso.


\section{An\'alisis de datos y conclusiones}
\subsection{Oscilador arm\'onico amortiguado}

El oscilador arm\'onico amotiguado es el caso generalizado de los osciladores arm\'onicos libres ya que su comportamiento contempla la disipaci\'on de energ\'ia pero no la presencia de fuerzas externas. Su comportamiento est\'a dado por (1).
Para obtener las soluciones de dicha ecuacion, se propone una con la forma $x(t) = e^{zt}$, con $z \in \mathbb{C}$ (luego $\dot{x}(t) = ze^{zt}$ y $\ddot{x}(t) = z^2e^{zt}$).

Reemplazando el $x(t)$ y sus derivadas en (1), se obtiene la siguiente ecuacion cuadratica para $z$:

$$
(z^2 + 2\gamma z + \omega_o^2)e^{zt}=0
$$

Esta ecuacion es igual a 0 si y solo si $(z^2 + 2\gamma z + \omega_o^2) = 0$. Las raices de este polinomio son:

$$
z_{1,2} = \frac{-2\gamma \pm \sqrt{4\gamma^2 - 4\omega_0^2}}{2} = -\gamma \pm \sqrt{\gamma^2 - \omega_0^2}
$$

Podemos distinguir tres casos que, a continuaci\'on, analizamos por separado.

\begin{itemize}
\item Si $\omega_0^2 > \gamma^2$, tenemos dos raices complejas
$$z_{1} = -\gamma + i\sqrt{\omega_0^2 - \gamma^2}$$  $$z_{2} = -\gamma - i\sqrt{\omega_0^2 - \gamma^2}$$
Sea $\omega_{1} = \sqrt{\omega_0^2 - \gamma^2}$. Reemplazando esta nueva frecuencia en la solucion propuesta y tomando su parte real, tenemos que
\begin{equation}
x(t) = e^{-\gamma t}(acos(\omega_{1}t) + bsin(\omega_{1}t))
\end{equation}

$$\dot{x}(t) = e^{-\gamma t}((\omega_{1}b - \gamma a)acos(\omega_{1}t) + (\omega_{1}a - \gamma b)bsin(\omega_{1}t))$$

donde $a$ y $b$ se determinan con las condiciones iniciales
$$ x(0) = a, \dot{x}(0) = \omega_{1}b - \gamma a$$
Luego, la ecuacion (3) nos queda
\begin{equation}
x(t) = e^{-\gamma t}(x(0)cos(\omega_{1}t) + \frac{\dot{x}(0) + \gamma x(0)}{\omega_{1}}sin(\omega_{1}t))
\end{equation}

Este movimiento corresponde a una oscilacion armonica de frecuencia $\omega_{1}$, diferente de la frecuencia natural $\omega_{0}$; y se denomina \textit{movimiento oscilatorio subamortiguado}.

\begin{figure}[H]
\begin{center}
\includegraphics[height=5cm]{oscilacion_subamortiguada.JPG}
\caption[width=5cm]{Movimiento oscilatorio subamortiguado.\newline Se observa como la curva esta modulada por un termino exponencial relacionado con la constante de amortiguamiento $\gamma$.}
\end{center}
\end{figure}

\item Si $\omega_0^2 < \gamma^2$, tenemos dos raices reales
$$z_{1} = -\gamma + \sqrt{\gamma^2 - \omega_0^2}$$  $$z_{2} = -\gamma - \sqrt{\gamma^2 - \omega_0^2}$$


En este caso, la ecuacion $x(t)$ nos quedara expresada en termino de cosenos y senos hiperbolicos. Para las condiciones iniciales procedemos como en el caso del oscilador subamortiguado, por lo tanto la ecuacion de movimiento nos queda

\begin{equation}
x(t) = x(0)e^{-\gamma t}cosh(\sqrt{\gamma^2 - \omega_0^2}t) + \frac{\dot{x}(0) + \gamma x(0)}{\sqrt{\gamma^2 - \omega_0^2}}e^{-\gamma t}sinh(\sqrt{\gamma^2 - \omega_0^2}t)
\end{equation}

Este movimiento se denomina \textit{oscilatorio sobreamortiguado}.

\begin{figure}[H]
\begin{center}
\includegraphics[height=5cm]{oscilacion_sobreamortiguada.JPG}
\caption[width=5cm]{Esquema del movimiento oscilatorio sobreamortiguado. En $t_{0}$ el sistema decae a 0 sin llegar a completar un periodo de oscilacion.}
\end{center}
\end{figure}


\item Si $\omega_0^2 = \gamma^2$, tenemos una raiz real doble

$$z_{1,2} = -\gamma$$

Al tener una unica raiz doble, debemos considerar soluciones del siguiente tipo: 

$$ x(t) = (a+bt)e^{-\gamma t}$$

Al imponer las condiciones iniciales sobre $x(t)$, la solucion nos queda que

\begin{equation}
x(t) = [x(0) + (\dot{x}(0) + \gamma x(0))t]e^{-\gamma t}
\end{equation}

Este movimiento se denomina \textit{oscilatorio con amortiguamiento critico}.

\begin{figure}[H]
\begin{center}
\includegraphics[height=5cm]{oscilacion_amortiguada_c.JPG}
\caption[width=5cm]{Esquema del movimiento oscilatorio con amortiguamiento critico}
\end{center}
\end{figure}

\end{itemize}





\subsection{Oscilador de Van der Pol}



\section{Ap\'endice}

Descargar el archivo $guia1\_poggi.zip$ y descomprimirlo en una carpeta a elecci\'on del usuario, abrir una terminal dentro de la carpeta $\sim \backslash mecanica-clasica\backslash practica1$; y seguir las instrucciones provistas en la secci\'on Enunciado.

Los archivos que contienen los datos num\'ericos $(ej2.dat$ y $ej3.dat)$ no fueron transcriptos en este informe dada la extensi\'on de los mismos


\subsection{C\'odigo fuente en C del oscilador arm\'onico amortiguado (ODE\_ej2.c)}

\lstinputlisting[language=C,breaklines]{../ODE_ej2.c}

\subsection{C\'odigo fuente en C del oscilador de Van der Pol (ODE\_ej3.c)}
aasdasdjkasdjk

\subsection{C\'odigo fuente en C del m\'etodo de Runge-Kutta de orden 4 provisto por la c\'atedra (rk4.c)}
\lstinputlisting[language=C,breaklines]{../rk4.c}

\section{Bibliograf\'ia}




\end{document}
